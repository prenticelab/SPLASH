% stash_appdx.tex
%
% written by Tyler W. Davis
% Imperial College London
%
% 2014-10-29 -- created
% 2014-10-29 -- last updated
%
% ------------
% description:
% ------------
% This TEX file contains the appendix of the STASH 2.0 code book.
%
% ----------
% changelog:
% ----------
% 01. modularized chapter [14.10.29]
% 02. newline for each sentence [14.10.29]
% --> simpler for Git version control
%
%% \\\\\\\\\\\\\\\\\\\\\\\\\\\\\\\\\\\\\\\\\\\\\\\\\\\\\\\\\\\\\\\\\\\\\\\\ %%
%% APPENDIX A -- PYTHON CODE SNIPPETS
%% //////////////////////////////////////////////////////////////////////// %%
\section{Python Code}

%% \\\\\\\\\\\\\\\\\\\\\\\\\\\\\\\\\\\\\\\\\\\\\\\\\\\\\\\\\\\\\\\\\\\\\\\\ %%
%% APPENDIX A.1 -- JULIAN DAY
%% //////////////////////////////////////////////////////////////////////// %%
\subsection{Julian Day}
\label{app:jday}
\texttt{ \\
\noindent 01~~def julian\textunderscore day(y, m, i):\\
\noindent 02 \indent if m <= 2:\\
\noindent 03 \indent \indent y = y-1\\
\noindent 04 \indent \indent m = m+12\\
\noindent 05 \indent A = int(y/100)\\
\noindent 06 \indent B = 2-A+int(A/4)\\
\noindent 07 \indent JDE = int(365.25*(y+4716))+int(30.6001*(m+1))+i+B-1524.5
\\
\noindent 08 \indent return(JDE)\\
}

\noindent This script calculates the Julian day for a given date in the Gregorian calendar \parencite[Ch. 7]{meeus91}, where \texttt{y} is the year, \texttt{m} is the month (i.e., 1--12), and \texttt{i} is the day (i.e., 1--31). 
The term $B$ on Line 06 is for the modified definition of leap-years in the Gregorian calendar from the Julian calendar. If using the Julian calendar dates, set $B$ equal to zero. 
To test whether the algorithm is correctly implemented, the Julian day of 13 Aug 2014 is 2456882.

\newpage

%% \\\\\\\\\\\\\\\\\\\\\\\\\\\\\\\\\\\\\\\\\\\\\\\\\\\\\\\\\\\\\\\\\\\\\\\\ %%
%% APPENDIX A.2 -- Berger's Method
%% //////////////////////////////////////////////////////////////////////// %%
\subsection{Berger's Method}
\label{app:berger}
\texttt{ \\
\noindent 01~~def berger\_tls(n):\\
\noindent 02 \indent pir = numpy.pi/180.0 \\
\noindent 03 \indent xee = ke**2 \\
\noindent 04 \indent xec = ke**3 \\
\noindent 05 \indent xse = numpy.sqrt(1.0 - xee) \\
\noindent 06 \indent \# Mean longitude: \\
\noindent 07 \indent xlam = ( \\
\noindent 08 \indent \indent (ke/2.0 + xec/8.0)*(1.0 + xse)*numpy.sin(komega*pir) -  \\
\noindent 09 \indent \indent xee/4.0*(0.5 + xse)*numpy.sin(2.0*komega*pir) +  \\
\noindent 10 \indent \indent xec/8.0*(1.0/3.0 + xse)*numpy.sin(3.0*komega*pir) \\
\noindent 11 \indent \indent ) \\
\noindent 12 \indent xlam = numpy.degrees(2.0*xlam) \\
\noindent 13 \indent \# Mean longitude for day of year: \\
\noindent 14 \indent dlamm = xlam + (n - 80.0)*(360.0/kN) \\
\noindent 15 \indent \# Mean anomaly: \\
\noindent 16 \indent anm = dlamm - komega \\
\noindent 17 \indent ranm = numpy.radians(anm) \\
\noindent 18 \indent \# True anomaly (uncorrected): \\
\noindent 19 \indent ranv = (ranm + (2.0*ke - xec/4.0)*numpy.sin(ranm) +  \\
\noindent 20 \indent \indent 5.0/4.0*xee*numpy.sin(2.0*ranm) +  \\
\noindent 21 \indent \indent 13.0/12.0*xec*numpy.sin(3.0*ranm)) \\
\noindent 22 \indent anv = numpy.degrees(ranv) \\
\noindent 23 \indent \# True longitude: \\
\noindent 24 \indent my\_tls = anv + komega \\
\noindent 25 \indent if my\_tls < 0: \\
\noindent 26 \indent \indent my\_tls += 360.0 \\
\noindent 28 \indent elif my\_tls > 360: \\
\noindent 29 \indent \indent my\_tls -= 360.0 \\
\noindent 30 \indent \# True anomaly: \\
\noindent 31 \indent my\_nu = (my\_tls - komega) \\
\noindent 32 \indent if my\_nu < 0: \\
\noindent 33 \indent \indent my\_nu += 360.0 \\
\noindent 34 \indent return(my\_nu, my\_tls) \\
}

This algorithm calculates the true anomaly and true longitude for a given day